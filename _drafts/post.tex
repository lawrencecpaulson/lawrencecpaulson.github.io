\documentclass[11pt]{article}

\begin{document}
Let $\{B_x\}_{x\in A}$ be a family of sets indexed by some set $A$ and consider the set of all functions $f$ such that $f(x)\in B_x$ for all $x\in A$. This set can be written $\prod_{x\in A}\,B_x$, the product of a family of sets. It is essentially the same concept as the product of a family of types, typically written $(\prod x:A)\,B(x)$ in type theory.

If some $B_x$ is empty then $f(x)\in B_x$ cannot hold and the product must be empty. Conversely, if $B_x$ is nonempty for all $x\in A$ that we expect the product to be nonempty. This natural assumption is actually the axiom of choice.

If you are wondering why an axiom is required, note that it claims the existence of a function satisfying quite specific properties. Even the existence of the set of natural numbers, which most of us take for granted, requires its own axiom: the axiom of infinity.

In the 19th century, mathematics was not done on an axiomatic basis and many eminent mathematicians made use of the axiom of choice without being aware that a principle was involved. Among the many facts equivalent to the axiom of choice is that a countable union of countable sets is countable.

When Zermelo promulgated the axiom of choice in 1908, the very sight of it or possibly Zermelo's application to proving that every set was well ordered aroused strong opposition. Among the fiercest opponents of the axiom were Baire and Lebesgue. Ironically, as we now know, the main results of both rely on choice for their significance.


\end{document}

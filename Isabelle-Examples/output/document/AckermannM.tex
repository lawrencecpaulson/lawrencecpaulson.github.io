%
\begin{isabellebody}%
\setisabellecontext{AckermannM}%
%
\isadelimdocument
%
\endisadelimdocument
%
\isatagdocument
%
\isamarkupsection{A Tail-Recursive, Stack-Based Ackermann's Function%
}
\isamarkuptrue%
%
\endisatagdocument
{\isafolddocument}%
%
\isadelimdocument
%
\endisadelimdocument
%
\begin{isamarkuptext}%
Unlike the other Ackermann example, this termination proof uses the argument from
Nachum Dershowitz and Zohar Manna. Proving termination with multiset orderings.
Communications of the ACM 22 (8) 1979, 465–476.%
\end{isamarkuptext}\isamarkuptrue%
%
\isadelimtheory
%
\endisadelimtheory
%
\isatagtheory
\isakeywordONE{theory}\isamarkupfalse%
\ AckermannM\ \isakeywordTWO{imports}\ {\isachardoublequoteopen}HOL{\isacharminus}{\kern0pt}Library{\isachardot}{\kern0pt}Multiset{\isacharunderscore}{\kern0pt}Order{\isachardoublequoteclose}\ {\isachardoublequoteopen}HOL{\isacharminus}{\kern0pt}Library{\isachardot}{\kern0pt}Product{\isacharunderscore}{\kern0pt}Lexorder{\isachardoublequoteclose}\isanewline
\isanewline
\isakeywordTWO{begin}%
\endisatagtheory
{\isafoldtheory}%
%
\isadelimtheory
%
\endisadelimtheory
%
\begin{isamarkuptext}%
This theory investigates a stack-based implementation of Ackermann's function.
Let's recall the traditional definition,
as modified by R{\'o}zsa P\'eter and Raphael Robinson.%
\end{isamarkuptext}\isamarkuptrue%
\isakeywordONE{fun}\isamarkupfalse%
\ ack\ {\isacharcolon}{\kern0pt}{\isacharcolon}{\kern0pt}\ {\isachardoublequoteopen}{\isacharbrackleft}{\kern0pt}nat{\isacharcomma}{\kern0pt}nat{\isacharbrackright}{\kern0pt}\ {\isasymRightarrow}\ nat{\isachardoublequoteclose}\ \isakeywordTWO{where}\isanewline
\ \ {\isachardoublequoteopen}ack\ {\isadigit{0}}\ n\ \ \ \ \ \ \ \ \ \ \ \ \ {\isacharequal}{\kern0pt}\ Suc\ n{\isachardoublequoteclose}\isanewline
{\isacharbar}{\kern0pt}\ {\isachardoublequoteopen}ack\ {\isacharparenleft}{\kern0pt}Suc\ m{\isacharparenright}{\kern0pt}\ {\isadigit{0}}\ \ \ \ \ \ \ {\isacharequal}{\kern0pt}\ ack\ m\ {\isadigit{1}}{\isachardoublequoteclose}\isanewline
{\isacharbar}{\kern0pt}\ {\isachardoublequoteopen}ack\ {\isacharparenleft}{\kern0pt}Suc\ m{\isacharparenright}{\kern0pt}\ {\isacharparenleft}{\kern0pt}Suc\ n{\isacharparenright}{\kern0pt}\ {\isacharequal}{\kern0pt}\ ack\ m\ {\isacharparenleft}{\kern0pt}ack\ {\isacharparenleft}{\kern0pt}Suc\ m{\isacharparenright}{\kern0pt}\ n{\isacharparenright}{\kern0pt}{\isachardoublequoteclose}%
\begin{isamarkuptext}%
Setting up the termination proof for the stack-based version.%
\end{isamarkuptext}\isamarkuptrue%
\isakeywordONE{fun}\isamarkupfalse%
\ ack{\isacharunderscore}{\kern0pt}mset\ {\isacharcolon}{\kern0pt}{\isacharcolon}{\kern0pt}\ {\isachardoublequoteopen}nat\ list\ {\isasymRightarrow}\ {\isacharparenleft}{\kern0pt}nat{\isasymtimes}nat{\isacharparenright}{\kern0pt}\ multiset{\isachardoublequoteclose}\ \isakeywordTWO{where}\isanewline
\ \ {\isachardoublequoteopen}ack{\isacharunderscore}{\kern0pt}mset\ {\isacharbrackleft}{\kern0pt}{\isacharbrackright}{\kern0pt}\ {\isacharequal}{\kern0pt}\ {\isacharbraceleft}{\kern0pt}{\isacharhash}{\kern0pt}{\isacharbraceright}{\kern0pt}{\isachardoublequoteclose}\isanewline
{\isacharbar}{\kern0pt}\ {\isachardoublequoteopen}ack{\isacharunderscore}{\kern0pt}mset\ {\isacharbrackleft}{\kern0pt}x{\isacharbrackright}{\kern0pt}\ {\isacharequal}{\kern0pt}\ {\isacharbraceleft}{\kern0pt}{\isacharhash}{\kern0pt}{\isacharbraceright}{\kern0pt}{\isachardoublequoteclose}\isanewline
{\isacharbar}{\kern0pt}\ {\isachardoublequoteopen}ack{\isacharunderscore}{\kern0pt}mset\ {\isacharparenleft}{\kern0pt}z{\isacharhash}{\kern0pt}y{\isacharhash}{\kern0pt}l{\isacharparenright}{\kern0pt}\ {\isacharequal}{\kern0pt}\ \ mset\ {\isacharparenleft}{\kern0pt}{\isacharparenleft}{\kern0pt}y{\isacharcomma}{\kern0pt}z{\isacharparenright}{\kern0pt}\ {\isacharhash}{\kern0pt}\ map\ {\isacharparenleft}{\kern0pt}{\isasymlambda}x{\isachardot}{\kern0pt}\ {\isacharparenleft}{\kern0pt}Suc\ x{\isacharcomma}{\kern0pt}\ {\isadigit{0}}{\isacharparenright}{\kern0pt}{\isacharparenright}{\kern0pt}\ l{\isacharparenright}{\kern0pt}{\isachardoublequoteclose}\isanewline
\isanewline
\isakeywordONE{lemma}\isamarkupfalse%
\ case{\isadigit{1}}{\isacharcolon}{\kern0pt}\ {\isachardoublequoteopen}ack{\isacharunderscore}{\kern0pt}mset\ {\isacharparenleft}{\kern0pt}Suc\ n\ {\isacharhash}{\kern0pt}\ l{\isacharparenright}{\kern0pt}\ {\isacharless}{\kern0pt}\ add{\isacharunderscore}{\kern0pt}mset\ {\isacharparenleft}{\kern0pt}{\isadigit{0}}{\isacharcomma}{\kern0pt}\ n{\isacharparenright}{\kern0pt}\ {\isacharbraceleft}{\kern0pt}{\isacharhash}{\kern0pt}{\isacharparenleft}{\kern0pt}Suc\ x{\isacharcomma}{\kern0pt}\ {\isadigit{0}}{\isacharparenright}{\kern0pt}{\isachardot}{\kern0pt}\ x\ {\isasymin}{\isacharhash}{\kern0pt}\ mset\ l{\isacharhash}{\kern0pt}{\isacharbraceright}{\kern0pt}{\isachardoublequoteclose}\isanewline
%
\isadelimproof
%
\endisadelimproof
%
\isatagproof
\isakeywordONE{proof}\isamarkupfalse%
\ {\isacharparenleft}{\kern0pt}cases\ l{\isacharparenright}{\kern0pt}\isanewline
\ \ \isakeywordTHREE{case}\isamarkupfalse%
\ {\isacharparenleft}{\kern0pt}Cons\ m\ list{\isacharparenright}{\kern0pt}\isanewline
\ \ \isakeywordONE{have}\isamarkupfalse%
\ {\isachardoublequoteopen}{\isacharbraceleft}{\kern0pt}{\isacharhash}{\kern0pt}{\isacharparenleft}{\kern0pt}m{\isacharcomma}{\kern0pt}\ Suc\ n{\isacharparenright}{\kern0pt}{\isacharhash}{\kern0pt}{\isacharbraceright}{\kern0pt}\ {\isacharless}{\kern0pt}\ {\isacharbraceleft}{\kern0pt}{\isacharhash}{\kern0pt}{\isacharparenleft}{\kern0pt}Suc\ m{\isacharcomma}{\kern0pt}\ {\isadigit{0}}{\isacharparenright}{\kern0pt}{\isacharhash}{\kern0pt}{\isacharbraceright}{\kern0pt}{\isachardoublequoteclose}\isanewline
\ \ \ \ \isakeywordONE{by}\isamarkupfalse%
\ auto\isanewline
\ \ \isakeywordONE{also}\isamarkupfalse%
\ \isakeywordONE{have}\isamarkupfalse%
\ {\isachardoublequoteopen}{\isasymdots}\ {\isasymle}\ {\isacharbraceleft}{\kern0pt}{\isacharhash}{\kern0pt}{\isacharparenleft}{\kern0pt}Suc\ m{\isacharcomma}{\kern0pt}\ {\isadigit{0}}{\isacharparenright}{\kern0pt}{\isacharcomma}{\kern0pt}\ {\isacharparenleft}{\kern0pt}{\isadigit{0}}{\isacharcomma}{\kern0pt}n{\isacharparenright}{\kern0pt}{\isacharhash}{\kern0pt}{\isacharbraceright}{\kern0pt}{\isachardoublequoteclose}\isanewline
\ \ \ \ \isakeywordONE{by}\isamarkupfalse%
\ auto\isanewline
\ \ \isakeywordONE{finally}\isamarkupfalse%
\ \isakeywordTHREE{show}\isamarkupfalse%
\ {\isacharquery}{\kern0pt}thesis\isanewline
\ \ \ \ \isakeywordONE{by}\isamarkupfalse%
\ {\isacharparenleft}{\kern0pt}simp\ add{\isacharcolon}{\kern0pt}\ Cons{\isacharparenright}{\kern0pt}\isanewline
\isakeywordONE{qed}\isamarkupfalse%
\ auto%
\endisatagproof
{\isafoldproof}%
%
\isadelimproof
%
\endisadelimproof
%
\begin{isamarkuptext}%
Here is the stack-based version, which uses lists.%
\end{isamarkuptext}\isamarkuptrue%
\isakeywordONE{function}\isamarkupfalse%
\ ackloop\ {\isacharcolon}{\kern0pt}{\isacharcolon}{\kern0pt}\ {\isachardoublequoteopen}nat\ list\ {\isasymRightarrow}\ nat{\isachardoublequoteclose}\ \isakeywordTWO{where}\isanewline
\ \ {\isachardoublequoteopen}ackloop\ {\isacharparenleft}{\kern0pt}n\ {\isacharhash}{\kern0pt}\ {\isadigit{0}}\ {\isacharhash}{\kern0pt}\ l{\isacharparenright}{\kern0pt}\ \ \ \ \ \ \ \ \ {\isacharequal}{\kern0pt}\ ackloop\ {\isacharparenleft}{\kern0pt}Suc\ n\ {\isacharhash}{\kern0pt}\ l{\isacharparenright}{\kern0pt}{\isachardoublequoteclose}\isanewline
{\isacharbar}{\kern0pt}\ {\isachardoublequoteopen}ackloop\ {\isacharparenleft}{\kern0pt}{\isadigit{0}}\ {\isacharhash}{\kern0pt}\ Suc\ m\ {\isacharhash}{\kern0pt}\ l{\isacharparenright}{\kern0pt}\ \ \ \ \ {\isacharequal}{\kern0pt}\ ackloop\ {\isacharparenleft}{\kern0pt}{\isadigit{1}}\ {\isacharhash}{\kern0pt}\ m\ {\isacharhash}{\kern0pt}\ l{\isacharparenright}{\kern0pt}{\isachardoublequoteclose}\isanewline
{\isacharbar}{\kern0pt}\ {\isachardoublequoteopen}ackloop\ {\isacharparenleft}{\kern0pt}Suc\ n\ {\isacharhash}{\kern0pt}\ Suc\ m\ {\isacharhash}{\kern0pt}\ l{\isacharparenright}{\kern0pt}\ {\isacharequal}{\kern0pt}\ ackloop\ {\isacharparenleft}{\kern0pt}n\ {\isacharhash}{\kern0pt}\ Suc\ m\ {\isacharhash}{\kern0pt}\ m\ {\isacharhash}{\kern0pt}\ l{\isacharparenright}{\kern0pt}{\isachardoublequoteclose}\isanewline
{\isacharbar}{\kern0pt}\ {\isachardoublequoteopen}ackloop\ {\isacharbrackleft}{\kern0pt}m{\isacharbrackright}{\kern0pt}\ {\isacharequal}{\kern0pt}\ m{\isachardoublequoteclose}\isanewline
{\isacharbar}{\kern0pt}\ {\isachardoublequoteopen}ackloop\ {\isacharbrackleft}{\kern0pt}{\isacharbrackright}{\kern0pt}\ {\isacharequal}{\kern0pt}\ \ {\isadigit{0}}{\isachardoublequoteclose}\isanewline
%
\isadelimproof
\ \ %
\endisadelimproof
%
\isatagproof
\isakeywordONE{by}\isamarkupfalse%
\ pat{\isacharunderscore}{\kern0pt}completeness\ auto%
\endisatagproof
{\isafoldproof}%
%
\isadelimproof
\isanewline
%
\endisadelimproof
\isanewline
\isakeywordONE{termination}\isamarkupfalse%
\isanewline
%
\isadelimproof
\ \ %
\endisadelimproof
%
\isatagproof
\isakeywordONE{by}\isamarkupfalse%
\ {\isacharparenleft}{\kern0pt}relation\ {\isachardoublequoteopen}inv{\isacharunderscore}{\kern0pt}image\ {\isacharbraceleft}{\kern0pt}{\isacharparenleft}{\kern0pt}x{\isacharcomma}{\kern0pt}y{\isacharparenright}{\kern0pt}{\isachardot}{\kern0pt}\ x{\isacharless}{\kern0pt}y{\isacharbraceright}{\kern0pt}\ ack{\isacharunderscore}{\kern0pt}mset{\isachardoublequoteclose}{\isacharparenright}{\kern0pt}\ {\isacharparenleft}{\kern0pt}auto\ simp{\isacharcolon}{\kern0pt}\ wf\ case{\isadigit{1}}{\isacharparenright}{\kern0pt}%
\endisatagproof
{\isafoldproof}%
%
\isadelimproof
%
\endisadelimproof
%
\begin{isamarkuptext}%
Unlike the other Ackermann theory, no extra function is needed to prove equivalence%
\end{isamarkuptext}\isamarkuptrue%
\isakeywordONE{lemma}\isamarkupfalse%
\ ackloop{\isacharunderscore}{\kern0pt}ack{\isacharcolon}{\kern0pt}\ {\isachardoublequoteopen}ackloop\ {\isacharparenleft}{\kern0pt}n\ {\isacharhash}{\kern0pt}\ m\ {\isacharhash}{\kern0pt}\ l{\isacharparenright}{\kern0pt}\ {\isacharequal}{\kern0pt}\ ackloop\ {\isacharparenleft}{\kern0pt}ack\ m\ n\ {\isacharhash}{\kern0pt}\ l{\isacharparenright}{\kern0pt}{\isachardoublequoteclose}\isanewline
%
\isadelimproof
\ \ %
\endisadelimproof
%
\isatagproof
\isakeywordONE{by}\isamarkupfalse%
\ {\isacharparenleft}{\kern0pt}induction\ m\ n\ arbitrary{\isacharcolon}{\kern0pt}\ l\ rule{\isacharcolon}{\kern0pt}\ ack{\isachardot}{\kern0pt}induct{\isacharparenright}{\kern0pt}\ auto%
\endisatagproof
{\isafoldproof}%
%
\isadelimproof
\isanewline
%
\endisadelimproof
\isanewline
\isakeywordONE{theorem}\isamarkupfalse%
\ ack{\isacharcolon}{\kern0pt}\ {\isachardoublequoteopen}ack\ m\ n\ {\isacharequal}{\kern0pt}\ ackloop\ {\isacharbrackleft}{\kern0pt}n{\isacharcomma}{\kern0pt}m{\isacharbrackright}{\kern0pt}{\isachardoublequoteclose}\isanewline
%
\isadelimproof
\ \ %
\endisadelimproof
%
\isatagproof
\isakeywordONE{by}\isamarkupfalse%
\ {\isacharparenleft}{\kern0pt}simp\ add{\isacharcolon}{\kern0pt}\ ackloop{\isacharunderscore}{\kern0pt}ack{\isacharparenright}{\kern0pt}%
\endisatagproof
{\isafoldproof}%
%
\isadelimproof
\isanewline
%
\endisadelimproof
%
\isadelimtheory
\isanewline
%
\endisadelimtheory
%
\isatagtheory
\isakeywordTWO{end}\isamarkupfalse%
%
\endisatagtheory
{\isafoldtheory}%
%
\isadelimtheory
%
\endisadelimtheory
%
\end{isabellebody}%
\endinput
%:%file=~/Internet/MachineLogic/Isabelle-Examples/AckermannM.thy%:%
%:%11=5%:%
%:%23=7%:%
%:%24=8%:%
%:%25=9%:%
%:%33=11%:%
%:%34=11%:%
%:%35=12%:%
%:%36=13%:%
%:%45=15%:%
%:%46=16%:%
%:%47=17%:%
%:%49=19%:%
%:%50=19%:%
%:%51=20%:%
%:%52=21%:%
%:%53=22%:%
%:%55=24%:%
%:%57=26%:%
%:%58=26%:%
%:%59=27%:%
%:%60=28%:%
%:%61=29%:%
%:%62=30%:%
%:%63=31%:%
%:%64=31%:%
%:%71=32%:%
%:%72=32%:%
%:%73=33%:%
%:%74=33%:%
%:%75=34%:%
%:%76=34%:%
%:%77=35%:%
%:%78=35%:%
%:%79=36%:%
%:%80=36%:%
%:%81=36%:%
%:%82=37%:%
%:%83=37%:%
%:%84=38%:%
%:%85=38%:%
%:%86=38%:%
%:%87=39%:%
%:%88=39%:%
%:%89=40%:%
%:%90=40%:%
%:%99=42%:%
%:%101=44%:%
%:%102=44%:%
%:%103=45%:%
%:%104=46%:%
%:%105=47%:%
%:%106=48%:%
%:%107=49%:%
%:%110=50%:%
%:%114=50%:%
%:%115=50%:%
%:%120=50%:%
%:%123=51%:%
%:%124=52%:%
%:%125=52%:%
%:%128=53%:%
%:%132=53%:%
%:%133=53%:%
%:%142=55%:%
%:%144=56%:%
%:%145=56%:%
%:%148=57%:%
%:%152=57%:%
%:%153=57%:%
%:%158=57%:%
%:%161=58%:%
%:%162=59%:%
%:%163=59%:%
%:%166=60%:%
%:%170=60%:%
%:%171=60%:%
%:%176=60%:%
%:%181=61%:%
%:%186=62%:%

%
\begin{isabellebody}%
\setisabellecontext{Numeric}%
%
\isadelimdocument
%
\endisadelimdocument
%
\isatagdocument
%
\isamarkupsection{Numerical experiments%
}
\isamarkuptrue%
%
\endisatagdocument
{\isafolddocument}%
%
\isadelimdocument
%
\endisadelimdocument
%
\isadelimtheory
%
\endisadelimtheory
%
\isatagtheory
\isakeywordONE{theory}\isamarkupfalse%
\ Numeric\ \isakeywordTWO{imports}\isanewline
\ \ {\isachardoublequoteopen}HOL{\isacharminus}{\kern0pt}Decision{\isacharunderscore}{\kern0pt}Procs{\isachardot}{\kern0pt}Approximation{\isachardoublequoteclose}\ {\isachardoublequoteopen}HOL{\isacharminus}{\kern0pt}Computational{\isacharunderscore}{\kern0pt}Algebra{\isachardot}{\kern0pt}Primes{\isachardoublequoteclose}\isanewline
\ \ \ \isanewline
\isakeywordTWO{begin}%
\endisatagtheory
{\isafoldtheory}%
%
\isadelimtheory
%
\endisadelimtheory
%
\begin{isamarkuptext}%
Addition of polymorphic numerals actually works, 
though nobody should rely on this%
\end{isamarkuptext}\isamarkuptrue%
\isakeywordONE{lemma}\isamarkupfalse%
\ {\isachardoublequoteopen}{\isadigit{2}}{\isacharplus}{\kern0pt}{\isadigit{2}}{\isacharequal}{\kern0pt}{\isadigit{4}}{\isachardoublequoteclose}\isanewline
%
\isadelimproof
\ \ %
\endisadelimproof
%
\isatagproof
\isakeywordONE{by}\isamarkupfalse%
\ auto%
\endisatagproof
{\isafoldproof}%
%
\isadelimproof
%
\endisadelimproof
%
\begin{isamarkuptext}%
Multiplication of polymorphic numerals does not work%
\end{isamarkuptext}\isamarkuptrue%
\isakeywordONE{lemma}\isamarkupfalse%
\ {\isachardoublequoteopen}{\isadigit{2}}{\isacharasterisk}{\kern0pt}{\isadigit{3}}{\isacharequal}{\kern0pt}{\isadigit{6}}{\isachardoublequoteclose}\isanewline
%
\isadelimproof
\ \ %
\endisadelimproof
%
\isatagproof
\isakeywordONE{oops}\isamarkupfalse%
%
\endisatagproof
{\isafoldproof}%
%
\isadelimproof
%
\endisadelimproof
%
\begin{isamarkuptext}%
These do not work because the group identity law is not available.%
\end{isamarkuptext}\isamarkuptrue%
\isakeywordONE{lemma}\isamarkupfalse%
\ {\isachardoublequoteopen}{\isadigit{0}}{\isacharplus}{\kern0pt}{\isadigit{2}}{\isacharequal}{\kern0pt}{\isadigit{2}}{\isachardoublequoteclose}\ {\isachardoublequoteopen}{\isadigit{1}}{\isacharasterisk}{\kern0pt}{\isadigit{3}}{\isacharequal}{\kern0pt}{\isadigit{3}}{\isachardoublequoteclose}\isanewline
%
\isadelimproof
\ \ %
\endisadelimproof
%
\isatagproof
\isakeywordONE{oops}\isamarkupfalse%
%
\endisatagproof
{\isafoldproof}%
%
\isadelimproof
%
\endisadelimproof
%
\begin{isamarkuptext}%
Works because of the type constraint. And multiplcation is fast!%
\end{isamarkuptext}\isamarkuptrue%
\isakeywordONE{lemma}\isamarkupfalse%
\ {\isachardoublequoteopen}{\isadigit{1}}{\isadigit{2}}{\isadigit{3}}{\isadigit{4}}{\isadigit{5}}{\isadigit{6}}{\isadigit{7}}{\isadigit{8}}{\isadigit{9}}\ {\isacharasterisk}{\kern0pt}\ {\isacharparenleft}{\kern0pt}{\isadigit{9}}{\isadigit{8}}{\isadigit{7}}{\isadigit{6}}{\isadigit{5}}{\isadigit{4}}{\isadigit{3}}{\isadigit{2}}{\isadigit{1}}{\isacharcolon}{\kern0pt}{\isacharcolon}{\kern0pt}int{\isacharparenright}{\kern0pt}\ {\isacharequal}{\kern0pt}\ {\isadigit{1}}{\isadigit{2}}{\isadigit{1}}{\isadigit{9}}{\isadigit{3}}{\isadigit{2}}{\isadigit{6}}{\isadigit{3}}{\isadigit{1}}{\isadigit{1}}{\isadigit{1}}{\isadigit{2}}{\isadigit{6}}{\isadigit{3}}{\isadigit{5}}{\isadigit{2}}{\isadigit{6}}{\isadigit{9}}{\isachardoublequoteclose}\isanewline
%
\isadelimproof
\ \ %
\endisadelimproof
%
\isatagproof
\isakeywordONE{by}\isamarkupfalse%
\ simp%
\endisatagproof
{\isafoldproof}%
%
\isadelimproof
%
\endisadelimproof
%
\begin{isamarkuptext}%
The function \isa{\isaconst{Suc}} implies type \isa{\isatconst{nat}}%
\end{isamarkuptext}\isamarkuptrue%
\isakeywordONE{lemma}\isamarkupfalse%
\ {\isachardoublequoteopen}Suc\ {\isacharparenleft}{\kern0pt}Suc\ {\isadigit{0}}{\isacharparenright}{\kern0pt}\ {\isacharasterisk}{\kern0pt}\ n\ {\isacharequal}{\kern0pt}\ n{\isacharasterisk}{\kern0pt}{\isadigit{2}}{\isachardoublequoteclose}\isanewline
%
\isadelimproof
\ \ %
\endisadelimproof
%
\isatagproof
\isakeywordONE{by}\isamarkupfalse%
\ simp%
\endisatagproof
{\isafoldproof}%
%
\isadelimproof
%
\endisadelimproof
%
\begin{isamarkuptext}%
We have to expand 5 into Suc-notation.%
\end{isamarkuptext}\isamarkuptrue%
\isakeywordONE{lemma}\isamarkupfalse%
\ {\isachardoublequoteopen}x{\isacharcircum}{\kern0pt}{\isadigit{5}}\ {\isacharequal}{\kern0pt}\ x{\isacharasterisk}{\kern0pt}x{\isacharasterisk}{\kern0pt}x{\isacharasterisk}{\kern0pt}x{\isacharasterisk}{\kern0pt}{\isacharparenleft}{\kern0pt}x{\isacharcolon}{\kern0pt}{\isacharcolon}{\kern0pt}real{\isacharparenright}{\kern0pt}{\isachardoublequoteclose}\isanewline
%
\isadelimproof
\ \ %
\endisadelimproof
%
\isatagproof
\isakeywordONE{by}\isamarkupfalse%
\ {\isacharparenleft}{\kern0pt}simp\ add{\isacharcolon}{\kern0pt}\ eval{\isacharunderscore}{\kern0pt}nat{\isacharunderscore}{\kern0pt}numeral{\isacharparenright}{\kern0pt}%
\endisatagproof
{\isafoldproof}%
%
\isadelimproof
%
\endisadelimproof
%
\begin{isamarkuptext}%
Decimal notation and arithmetic on complex numbers%
\end{isamarkuptext}\isamarkuptrue%
\isakeywordONE{lemma}\isamarkupfalse%
\ {\isachardoublequoteopen}{\isacharparenleft}{\kern0pt}{\isadigit{1}}\ {\isacharminus}{\kern0pt}\ {\isadigit{0}}{\isachardot}{\kern0pt}{\isadigit{3}}{\isacharasterisk}{\kern0pt}{\isasymi}{\isacharparenright}{\kern0pt}\ {\isacharasterisk}{\kern0pt}\ {\isacharparenleft}{\kern0pt}{\isadigit{2}}{\isachardot}{\kern0pt}{\isadigit{7}}\ {\isacharplus}{\kern0pt}\ {\isadigit{5}}{\isacharasterisk}{\kern0pt}{\isasymi}{\isacharparenright}{\kern0pt}\ {\isacharequal}{\kern0pt}\ {\isadigit{4}}{\isachardot}{\kern0pt}{\isadigit{2}}\ {\isacharplus}{\kern0pt}\ {\isadigit{4}}{\isachardot}{\kern0pt}{\isadigit{1}}{\isadigit{9}}{\isacharasterisk}{\kern0pt}{\isasymi}{\isachardoublequoteclose}\isanewline
%
\isadelimproof
\ \ %
\endisadelimproof
%
\isatagproof
\isakeywordONE{by}\isamarkupfalse%
\ {\isacharparenleft}{\kern0pt}simp\ add{\isacharcolon}{\kern0pt}\ algebra{\isacharunderscore}{\kern0pt}simps{\isacharparenright}{\kern0pt}%
\endisatagproof
{\isafoldproof}%
%
\isadelimproof
%
\endisadelimproof
%
\begin{isamarkuptext}%
Applying a function to a numeral argument via eval. 
      But only if this has been set up.%
\end{isamarkuptext}\isamarkuptrue%
\isakeywordONE{lemma}\isamarkupfalse%
\ {\isachardoublequoteopen}fact\ {\isadigit{2}}{\isadigit{0}}\ {\isacharless}{\kern0pt}\ {\isacharparenleft}{\kern0pt}{\isadigit{2}}{\isadigit{4}}{\isadigit{3}}{\isadigit{2}}{\isadigit{9}}{\isadigit{0}}{\isadigit{2}}{\isadigit{0}}{\isadigit{0}}{\isadigit{8}}{\isadigit{1}}{\isadigit{7}}{\isadigit{6}}{\isadigit{6}}{\isadigit{4}}{\isadigit{0}}{\isadigit{0}}{\isadigit{0}}{\isadigit{1}}{\isacharcolon}{\kern0pt}{\isacharcolon}{\kern0pt}nat{\isacharparenright}{\kern0pt}{\isachardoublequoteclose}\isanewline
%
\isadelimproof
\ \ %
\endisadelimproof
%
\isatagproof
\isakeywordONE{by}\isamarkupfalse%
\ eval%
\endisatagproof
{\isafoldproof}%
%
\isadelimproof
%
\endisadelimproof
%
\begin{isamarkuptext}%
Testing primality via eval, takes a couple of seconds. 
      The type constraint is necessary!%
\end{isamarkuptext}\isamarkuptrue%
\isakeywordONE{lemma}\isamarkupfalse%
\ {\isachardoublequoteopen}prime\ {\isacharparenleft}{\kern0pt}{\isadigit{1}}{\isadigit{7}}{\isadigit{9}}{\isadigit{4}}{\isadigit{2}}{\isadigit{4}}{\isadigit{6}}{\isadigit{7}}{\isadigit{3}}{\isacharcolon}{\kern0pt}{\isacharcolon}{\kern0pt}nat{\isacharparenright}{\kern0pt}{\isachardoublequoteclose}\isanewline
%
\isadelimproof
\ \ %
\endisadelimproof
%
\isatagproof
\isakeywordONE{by}\isamarkupfalse%
\ eval%
\endisatagproof
{\isafoldproof}%
%
\isadelimproof
%
\endisadelimproof
%
\begin{isamarkuptext}%
A simple demonstration of the approximation method%
\end{isamarkuptext}\isamarkuptrue%
\isakeywordONE{lemma}\isamarkupfalse%
\ {\isachardoublequoteopen}{\isasymbar}pi\ {\isacharminus}{\kern0pt}\ {\isadigit{3}}{\isadigit{5}}{\isadigit{5}}{\isacharslash}{\kern0pt}{\isadigit{1}}{\isadigit{1}}{\isadigit{3}}{\isasymbar}\ {\isacharless}{\kern0pt}\ {\isadigit{1}}{\isacharslash}{\kern0pt}{\isadigit{1}}{\isadigit{0}}{\isacharcircum}{\kern0pt}{\isadigit{6}}{\isachardoublequoteclose}\isanewline
%
\isadelimproof
\ \ %
\endisadelimproof
%
\isatagproof
\isakeywordONE{by}\isamarkupfalse%
\ {\isacharparenleft}{\kern0pt}approximation\ {\isadigit{2}}{\isadigit{5}}{\isacharparenright}{\kern0pt}%
\endisatagproof
{\isafoldproof}%
%
\isadelimproof
%
\endisadelimproof
%
\begin{isamarkuptext}%
Ditto, the approximation method%
\end{isamarkuptext}\isamarkuptrue%
\isakeywordONE{lemma}\isamarkupfalse%
\ {\isachardoublequoteopen}{\isasymbar}sqrt\ {\isadigit{2}}\ {\isacharminus}{\kern0pt}\ {\isadigit{1}}{\isachardot}{\kern0pt}{\isadigit{4}}{\isadigit{1}}{\isadigit{4}}{\isadigit{2}}{\isadigit{1}}{\isadigit{3}}{\isadigit{5}}{\isadigit{6}}{\isadigit{2}}{\isadigit{4}}{\isasymbar}\ {\isacharless}{\kern0pt}\ {\isadigit{1}}{\isacharslash}{\kern0pt}{\isadigit{1}}{\isadigit{0}}{\isacharcircum}{\kern0pt}{\isadigit{1}}{\isadigit{0}}{\isachardoublequoteclose}\isanewline
%
\isadelimproof
\ \ %
\endisadelimproof
%
\isatagproof
\isakeywordONE{by}\isamarkupfalse%
\ {\isacharparenleft}{\kern0pt}approximation\ {\isadigit{3}}{\isadigit{5}}{\isacharparenright}{\kern0pt}%
\endisatagproof
{\isafoldproof}%
%
\isadelimproof
%
\endisadelimproof
%
\begin{isamarkuptext}%
The approximation method on a *closed* interval (SLOW). Must avoid zero!%
\end{isamarkuptext}\isamarkuptrue%
\isakeywordONE{lemma}\isamarkupfalse%
\isanewline
\ \ \isakeywordTWO{fixes}\ x{\isacharcolon}{\kern0pt}{\isacharcolon}{\kern0pt}real\isanewline
\ \ \isakeywordTWO{assumes}\ {\isachardoublequoteopen}x\ {\isasymin}\ {\isacharbraceleft}{\kern0pt}{\isadigit{0}}{\isachardot}{\kern0pt}{\isadigit{1}}\ {\isachardot}{\kern0pt}{\isachardot}{\kern0pt}\ {\isadigit{1}}{\isacharbraceright}{\kern0pt}{\isachardoublequoteclose}\isanewline
\ \ \isakeywordTWO{shows}\ {\isachardoublequoteopen}x\ {\isacharasterisk}{\kern0pt}\ ln{\isacharparenleft}{\kern0pt}x{\isacharparenright}{\kern0pt}\ {\isasymge}\ {\isacharminus}{\kern0pt}{\isadigit{0}}{\isachardot}{\kern0pt}{\isadigit{3}}{\isadigit{6}}{\isadigit{8}}{\isachardoublequoteclose}\isanewline
%
\isadelimproof
\ \ %
\endisadelimproof
%
\isatagproof
\isakeywordONE{using}\isamarkupfalse%
\ assms\ \isakeywordONE{by}\isamarkupfalse%
\ {\isacharparenleft}{\kern0pt}approximation\ {\isadigit{1}}{\isadigit{7}}\ splitting{\isacharcolon}{\kern0pt}\ x{\isacharequal}{\kern0pt}{\isadigit{1}}{\isadigit{3}}{\isacharparenright}{\kern0pt}%
\endisatagproof
{\isafoldproof}%
%
\isadelimproof
%
\endisadelimproof
%
\begin{isamarkuptext}%
A little more accuracy makes it MUCH slower 
  (the exact answer is -1/e = 0.36787944117144233)%
\end{isamarkuptext}\isamarkuptrue%
\isakeywordONE{lemma}\isamarkupfalse%
\isanewline
\ \ \isakeywordTWO{fixes}\ x{\isacharcolon}{\kern0pt}{\isacharcolon}{\kern0pt}real\isanewline
\ \ \isakeywordTWO{assumes}\ {\isachardoublequoteopen}x\ {\isasymin}\ {\isacharbraceleft}{\kern0pt}{\isadigit{0}}{\isachardot}{\kern0pt}{\isadigit{1}}\ {\isachardot}{\kern0pt}{\isachardot}{\kern0pt}\ {\isadigit{1}}{\isacharbraceright}{\kern0pt}{\isachardoublequoteclose}\isanewline
\ \ \isakeywordTWO{shows}\ {\isachardoublequoteopen}x\ {\isacharasterisk}{\kern0pt}\ ln{\isacharparenleft}{\kern0pt}x{\isacharparenright}{\kern0pt}\ {\isasymge}\ {\isacharminus}{\kern0pt}{\isadigit{0}}{\isachardot}{\kern0pt}{\isadigit{3}}{\isadigit{6}}{\isadigit{7}}{\isadigit{9}}{\isachardoublequoteclose}\isanewline
%
\isadelimproof
%
\endisadelimproof
%
\isatagproof
\isakeywordONE{using}\isamarkupfalse%
\ assms\isanewline
\ \ \isakeywordONE{by}\isamarkupfalse%
\ {\isacharparenleft}{\kern0pt}approximation\ {\isadigit{1}}{\isadigit{8}}\ splitting{\isacharcolon}{\kern0pt}\ x{\isacharequal}{\kern0pt}{\isadigit{1}}{\isadigit{6}}{\isacharparenright}{\kern0pt}%
\endisatagproof
{\isafoldproof}%
%
\isadelimproof
\ \isanewline
%
\endisadelimproof
%
\isadelimtheory
\isanewline
%
\endisadelimtheory
%
\isatagtheory
\isakeywordTWO{end}\isamarkupfalse%
%
\endisatagtheory
{\isafoldtheory}%
%
\isadelimtheory
%
\endisadelimtheory
%
\end{isabellebody}%
\endinput
%:%file=~/Internet/MachineLogic/Isabelle-Examples/Numeric.thy%:%
%:%11=1%:%
%:%27=3%:%
%:%28=3%:%
%:%29=4%:%
%:%30=5%:%
%:%31=6%:%
%:%40=8%:%
%:%41=9%:%
%:%43=10%:%
%:%44=10%:%
%:%47=11%:%
%:%51=11%:%
%:%52=11%:%
%:%61=13%:%
%:%63=14%:%
%:%64=14%:%
%:%67=15%:%
%:%71=15%:%
%:%81=17%:%
%:%83=18%:%
%:%84=18%:%
%:%87=19%:%
%:%91=19%:%
%:%101=21%:%
%:%103=22%:%
%:%104=22%:%
%:%107=23%:%
%:%111=23%:%
%:%112=23%:%
%:%121=25%:%
%:%123=26%:%
%:%124=26%:%
%:%127=27%:%
%:%131=27%:%
%:%132=27%:%
%:%141=29%:%
%:%143=30%:%
%:%144=30%:%
%:%147=31%:%
%:%151=31%:%
%:%152=31%:%
%:%161=33%:%
%:%163=34%:%
%:%164=34%:%
%:%167=35%:%
%:%171=35%:%
%:%172=35%:%
%:%181=37%:%
%:%182=38%:%
%:%184=39%:%
%:%185=39%:%
%:%188=40%:%
%:%192=40%:%
%:%193=40%:%
%:%202=42%:%
%:%203=43%:%
%:%205=44%:%
%:%206=44%:%
%:%209=45%:%
%:%213=45%:%
%:%214=45%:%
%:%223=47%:%
%:%225=48%:%
%:%226=48%:%
%:%229=49%:%
%:%233=49%:%
%:%234=49%:%
%:%243=51%:%
%:%245=52%:%
%:%246=52%:%
%:%249=53%:%
%:%253=53%:%
%:%254=53%:%
%:%263=55%:%
%:%265=56%:%
%:%266=56%:%
%:%267=57%:%
%:%268=58%:%
%:%269=59%:%
%:%272=60%:%
%:%276=60%:%
%:%277=60%:%
%:%278=60%:%
%:%287=62%:%
%:%288=63%:%
%:%290=64%:%
%:%291=64%:%
%:%292=65%:%
%:%293=66%:%
%:%294=67%:%
%:%301=68%:%
%:%302=68%:%
%:%303=69%:%
%:%304=69%:%
%:%309=69%:%
%:%314=70%:%
%:%319=71%:%
